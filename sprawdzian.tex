\documentclass[10pt,a4paper]{article}
\usepackage[utf8]{inputenc}
\usepackage[T1]{fontenc}
\usepackage{amsmath}
\usepackage{polski}
\usepackage{amsfonts}
\usepackage{amssymb}
\usepackage{graphicx}
\begin{document}
	
	\textbf{Zad 2}
	
	1.Wygenerowałem 2 klucze poprzez użycie komendy ssh-keygen. Klucz na serwer zapisałem w lokalizacji odmyślnej a klucz do repozytorium w innym pliku.
	
	2. Następnie skopiowałem klucz ssh dzięki komendzie ssh-copy-id na serwer uczelniany
	
	3.Stowrzyłem repozytorum i dodałem klucz ssh kopijąc go z pliku i dodając na githuba
	
	4. Stworzyłem plik config w folderze ./ssh w pliku podałem alias(pwi-sprawdzian) oraz użytkownika i ip serwera
	
	5.
	
	\textbf{Zad 3}
	
	1.
	
	2. pobieram za pomocą polecenia wget -r (--recursive) -np (--no-parent) 
	
	3. użycie komendy echo -n m324086 | md5sum wypisało zaszyfrowanego stringa następnie poprzez komendę find -name 'wynik z echo' znalazłem plik. Wszedłem do niego i następnie wykonałem 2 razy polecenie grep -wc (ma zliczyć ilośc wystopień podanego słowa) jako argumenty dałem 'POLAND' żeby zliczyło mi polskie konta oraz 'Country' żeby zliczyło mi wszystkie konta.
	\textbf{Wzory}
	 $ p\frac{D\textbf{u}}{Dt} = p(\frac{\delta\textbf{u}}{\delta t}+\textbf{u}\dot)$
		
\end{document}
